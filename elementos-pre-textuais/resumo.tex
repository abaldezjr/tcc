O presente trabalho de graduação apresenta o projeto de um sistema de coordenadas cartesianas para 
posicionamento de instrumentos de medição situado no túnel de vento do laboratório da Universidade 
Federal do Rio Grande. A concepção do projeto partiu da necessidade de precisão e repetibilidade  
no deslocamento  e posicionamento de instrumentos de medição  que caracterizam  o campo de velocidade 
e ou pressão na seção de teste do canal aerodinâmico. Para o desenvolvimento deste sistema, foram 
utilizadas plataformas de hardware livre, tais como o arduino, visando o baixo custo e fácil 
reprodução. O sistema consiste em um aplicativo que receberá a coordenada para onde a mesa deve 
se movimentar. O projeto pode ser dividido em três partes: mecânica, na qual envolve a disposição 
dos componentes mecânicos; elétrica, na qual se projetou os circuitos elétricos; e a programação, 
na qual foi desenvolvido o sistema de comando do aplicativo e a mesa. Logo, para a montagem do sistema 
eletrônico foram utilizados uma placa controladora  Arduino Uno, drivers de potência, motores de passo, 
optoacopladores, encoders. Já para o sistema mecânico foram utilizados eixos, mancais, rolamentos, 
fusos e  estrutura base/pórtico em alumínio.

