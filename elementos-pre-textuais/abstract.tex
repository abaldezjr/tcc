The present graduation work presents the project of a Cartesian coordinate system for positioning 
measuring instruments located in the wind tunnel of the laboratory of the Federal University of 
Rio Grande. The design of the project started from the need for precision and repeatability in 
the displacement and positioning of measuring instruments that characterize the speed and / or pressure 
field in the test section of the streamline. For the development of this system, free hardware platforms 
were used, such as arduino, aiming at low cost and easy reproduction. \textbf{O sistema consiste em um aplicativo 
que receberá a coordenada para onde a mesa deve se movimentar. O projeto pode ser dividido em três partes: 
mecânica, na qual envolve a disposição dos componentes mecânicos; elétrica, na qual se projetou os 
circuitos elétricos; e a programação, na qual foi desenvolvido o sistema de comando do aplicativo e 
a mesa. Logo, para a montagem do sistema eletrônico foram utilizados uma placa controladora 
Arduino Uno, drivers de potência, motores de passo, optoacopladores, encoders. Já para o sistema 
mecânico foram utilizados eixos, mancais, rolamentos, fusos e  estrutura base/pórtico em alumínio.}

\keywords{Túnel de vento. Tubo de Pitot. Arduino. Mesa cartesiana}
