\definecolor{codegreen}{rgb}{0,0.6,0}
\definecolor{codegray}{rgb}{0.5,0.5,0.5}
\definecolor{codepurple}{rgb}{0.58,0,0.82}
\definecolor{codeblack}{rgb}{0.0,0.0,0.0}
\definecolor{backcolour}{rgb}{1.00,1.00,1.00}

\lstdefinestyle{mystyle}{
    backgroundcolor     = \color{backcolour},   
    basicstyle          = \footnotesize,
    breakatwhitespace   = false,         
    breaklines          = true,                 
    captionpos          = b,                    
    commentstyle        = \color{codeblack},
    frame               = single,	
    keepspaces          = true,                 
    keywordstyle        = \textbf,
    language            = C++,
    numberstyle         = \tiny\color{codeblack},
    numbers             = left,                    
    numbersep           = 5pt,                  
    showspaces          = false,                
    showstringspaces    = false,
    showtabs            = false,                  
    stringstyle         = \color{codeblack},    
    tabsize             = 2
}

\apendice{CÓDIGO PRINCIPAL PRESENTE NO ARDUINO PARA O CONTROLE DA MESA CARTESIANA}
\label{ap:MesaCartesiana}

\lstset{
    caption         = {Código principal presente no arduino para o controle da mesa cartesiana.}, 
    label           = code:MesaCartesiana,
    language        = C++,
    morekeywords    = {Driver, Eixo, Pino, Sigmoidal},
    style           = mystyle,
}
{\lstinputlisting[language=C++]{elementos-pos-textuais/apendices/MesaCartesiana.txt}

\apendice{CÓDIGO DA CLASSE EIXO PRESENTE NO ARDUINO PARA O CONTROLE DO EIXO}
\label{ap:Eixo}

\lstset{
    caption         = {Código da classe driver presente no arduino para o controle do eixo.},
    label           = code:Eixo,
    language        = C++,
    morekeywords    = {Driver, Eixo, Pino, Sigmoidal},
    style           = mystyle,
}
\lstinputlisting[language=C++]{elementos-pos-textuais/apendices/Eixo.txt}

\apendice{CÓDIGO DA CLASSE DRIVER PRESENTE NO ARDUINO PARA O CONTROLE DO DRIVER DE POTÊNCIA.}
\label{ap:Driver}

\lstset{
    caption         = {Código da classe driver presente no arduino para o controle do driver de potência},
    label           = code:Driver,
    language        = C++,
    morekeywords    = {Driver, Eixo, Pino, Sigmoidal},
    style           = mystyle,
}
\lstinputlisting[language=C++]{elementos-pos-textuais/apendices/Driver.txt}

\apendice{CÓDIGO DA CLASSE PINO PRESENTE NO ARDUINO PARA O CONTROLE DOS PINOS}
\label{ap:Pino}

\lstset{
    caption         = {Código da classe Pino presente no arduino para o controle dos pinos.},
    label           = code:Pino,
    language        = C++,
    morekeywords    = {Driver, Eixo, Pino, Sigmoidal},
    style           = mystyle,
}
\lstinputlisting[language=C++]{elementos-pos-textuais/apendices/Pino.txt}

\apendice{CÓDIGO DA CLASSE SIGMOIDAL PRESENTE NO ARDUINO PARA O CONTROLE DA ACELERAÇÃO DOS MOTORES DE PASSO}
\label{ap:Sigmoidal}

\lstset{
    caption         = {Código da classe Sigmoidal presente no arduino para o controle da aceleração dos motores de passo.},
    label           = code:Sigmoidal,
    language        = C++,
    morekeywords    = {Driver, Eixo, Pino, Sigmoidal},
    style           = mystyle,
}
\lstinputlisting[language=C++]{elementos-pos-textuais/apendices/Sigmoidal.txt}

\apendice{CÓDIGO DO HEADER STDRIVER PRESENTE NO ARDUINO PARA O INCLUDE DE CLASSES}
\label{ap:StepperDriver}

\lstset{
    caption         = {Código do header StDriver presente no arduino para o include de classes.},
    label           = code:StDriver,
    language        = C++,
    morekeywords    = {Driver, Eixo, Pino, Sigmoidal},
    style           = mystyle,
}
\lstinputlisting[language=C++]{elementos-pos-textuais/apendices/StDriver.txt}