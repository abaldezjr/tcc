\chapter{Considerações finais}\label{chap:conclusao}

O objetivo geral do projeto que consiste na movimentação de instrumentos de medição em um sistema 
de coordenadas cartesianas foi atendido. Nesse sentido, os objetivos específicos de projetar a 
mesa cartesiana, criar o sistema de comunicação “mesa-software” e desenvolver o software 
de comando foram contemplados.

A mesa cartesiana foi projetada com recursos disponíveis do mercado indicando que é possível 
construir sistemas com propósitos parecidos que atendem solicitações de laboratórios.

A documentação deste trabalho foi realizada a fim de facilitar o entendimento 
para o desenvolvimento de outros projetos parecidos e assim estender o impacto deste 
trabalho em outras instituições de ensino. Isso permitirá que um maior número de pesquisadores 
possam realizar a captura de dados para experimentos de forma automatizada em túneis de vento. 

Assim, o código fonte presente na placa de prototipação Arduino está disponível 
em um repositório online no endereço <https://github.com/abaldezjr/MesaCartesiana>. 
Também está disponível no endereço <https://github.com/abaldezjr/tcc>, uma versão em Latex com todo o 
material necessário para a montagem do trabalho, este tem o objetivo de auxiliar a montagem de 
trabalhos de conclusão de curso futuros.

%Com este trabalho, foi possível utilizar diversos dos conhecimentos adquiridos ao longo do curso de 
%Engenharia Mecânica utilizando-os de forma integrada, visto que envolveu a concepção de um sistema 
%mecânico movimentado por motores elétricos, além de aproveitar-se do desenvolvimento de software 
%na integração do projeto como um todo.

\section{Sugestões para trabalhos futuros}\label{sec:sugestoes}

Algumas melhorias podem ser sugeridas para uma próxima versão do projeto como:

\begin{itemize}
\item Capturar os dados de forma digital através de um sensor.
\item Automatizar a captura de dados para que seja possível o usuário planejar uma rotina e saber o seu tempo estimado.
\item Integrar o sistema de software da mesa cartesiana a um aplicativo que faça geração de relatórios, para uma visualização dos dados de forma gráfica.
\item Desenvolver o circuito eletrônico em uma placa de circuito impresso com os devidos componentes.
\item Projetar um case (caixa) para abrigar os componentes do circuito eletrônico.
\end{itemize}