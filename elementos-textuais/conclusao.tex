\chapter{Considerações finais}
\label{chap:conclusao}

O objetivo geral do projeto que consiste na movimentação de instrumentos de medição em um sistema 
de coordenadas cartesianas foi atendido. Nesse sentido, os objetivos específicos de projetar a mesa cartesiana, criar o sistema de comunicação “mesa-software” e desenvolver o software de comando foram contemplados.
Primeiramente, para a criação da mesa cartesiana foi projetada uma estrutura mecânica em formato TAL.
Posteriormente, para a comunicação entre a mesa e o software foi desenvolvido TAL
Por fim, para o software que comandará a mesa foi desenvolvido TAL
Além disso, os testes experimentais de desempenho quanto ao posicionamento e aceleração mostraram que TAL
A documentação deste trabalho foi realizada a fim de facilitar o entendimento para o desenvolvimento de 
outros projetos parecidos e assim estender o impacto deste trabalho em outras instituições de ensino. 
Isso permitirá que um maior número de pesquisadores possam realizar a captura de dados para experimentos 
de forma automatizada em túneis de vento. O projeto está disponível para montagem em um repositório online 
no endereço <https://github.com/abaldezjr/MesaCartesiana>.
Com este trabalho, foi possível utilizar diversos dos conhecimentos adquiridos ao longo do curso de 
Engenharia Mecânica utilizando-os de forma integrada, visto que envolveu a concepção de um sistema 
mecânico movimentado por motores elétricos, além de aproveitar-se do desenvolvimento de software 
na integração do projeto como um todo.

\section{Críticas e sugestões de trabalhos futuros}\label{sec:criticas}

Algumas melhorias podem ser sugeridas para uma próxima versão do projeto como:

\begin{itemize}
    \item LALALALA
    \item LALALALA
    \item LALALALA
\end{itemize}