\chapter{Resultados e discussão}
\label{chap:resultados}

Para apresentação dos resultados encontrados ao longo da implantação da mesa cartesiana no 
laboratório de sistemas térmicos e uma posterior discussão, os temas foram divididos de acordo 
com a metodologia (seção 3). Ou seja, primeiramente é apresentado os resultados do sistema mecânico, 
posteriormente os resultados do sistema eletrônico, em seguida é mostrado os resultados da programação 
do sistema de software. Por fim, é realizada uma análise dos testes experimentais.

\section{Sistema mecânico}

A Figura X mostra uma fotografia da mesa cartesiana possibilitando uma melhor compreensão do funcionamento 
dos seus componentes.

%
%
%COLOCAR IMAGEM DA MESA CARTESIANA

\section{Sistema eletrônico}

A Figura X mostra um esquema elétrico que comanda os motores de passos presentes na estrutura  da mesa 
cartesiana possibilitando uma melhor compreensão do funcionamento dos seus componentes.

%
%
%COLOCAR A IMAGEM DO ESQUEMA ELÉTRICO AQUI QUE DA PARA FAZER PELO TINKERCAD.COM


\section{Sistema de software}

O sistema de software do projeto foi fundamentado na metodologia, sendo assim para o desenvolvimento do código 
foi utilizado a plataforma de prototipação Arduino IDE respeitando o diagrama de classes mostrado na seção 3.3.3.


\section{Análise dos testes experimentais}

Os testes experimentais foram divididos em etapas. Primeiramente foi verificado o funcionamento dos sistemas 
separadamente. Sendo assim, foi verificado se a estrutura mecânica estava bem montada. Posteriormente foi verificado 
o funcionamento dos componentes eletrônicos, em seguida foi desenvolvido um código fonte simplificado para verificar 
os movimentos dos eixos. Por fim, o código fonte desenvolvido foi carregado no Arduino para que testes de desempenho 
fossem realizados.
Assim, para testar o desempenho da mesa cartesiana desenvolvida, foram definidos testes ao posicionamento e a 
aceleração para validação do projeto. No teste ao posicionamento, a mesa percorre alguns pontos distintos e em 
seguida é verificado o deslocamento nos dois eixos com um paquímetro. Já no teste a aceleração, ……………...

\subsection{Teste ao posicionamento}

Os motores de passo não possuem a capacidade de quantização de distâncias, dessa forma para garantir que a
posição da mesa corresponde a posição pretendida se faz o uso do encoder.
A origem é o único ponto da mesa conhecido pelo software, pois é possível através das chaves 
fim de curso impedir a movimentação dos motores, essa lógica de chaves fim de curso é aproveitada 
na inicialização do programa para a setagem da coordenada (0,0). Após a inicialização é possível que 
alguns erros aconteçam como: algum erro no software ou falha no hardware ocorrendo uma descalibração da mesa.
Assim, para validar a funcionalidade do software e hardware foi efetuado o seguinte teste:

\begin{alineas}
    \item Cinco coordenadas (x,y) foram estipuladas incluindo a coordenada mínima e máxima da mesa onde estavam 
    presentes as chaves de início e fim de curso.
    \item A medida que uma coordenada foi executada pela mesa, o deslocamento foi verificado pelo paquímetro.
    \item Após o teste terminar, o regresso a coordenada (0,0) foi efetuado para repetição do processo.
\end{alineas}

% Tabela z
% Resultados das medições executadas
% (Fonte: Próprio autor.)

A tabela X mostra que não existe diferença mensurável entre a posição pretendida e a medida comprovando 
a funcionalidade do sistema, bem como a precisão dos motores de passo.

\subsection{Teste a aceleração}
Os motores de passo funcionam através do envio de pulsos em um determinado intervalo de tempo, sendo assim 
conforme o intervalo de tempo que é enviado esse pulso elétrico, a velocidade do motor é modificada. Então, 
se conclui que:

\begin{alineas}
    \item Para um intervalo de tempo fixo, a velocidade do motor é constante.
    \item Para um intervalo de tempo que diminui ao longo do tempo, a velocidade aumenta. (Aceleração). 
    \item Para um intervalo de tempo que aumenta ao longo do tempo, a velocidade diminui. (Desaceleração).
    \item Para realizar a variação no intervalo de tempo para o envio de pulsos elétricos, foi escolhida 
    uma curva sigmóide que é desenhada em um gráfico de Período x Iterações
\end{alineas}

CONTINUAR EXPLICANDO