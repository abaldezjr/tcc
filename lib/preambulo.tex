\usepackage[utf8]{inputenc}                         % Acentuação direta
\usepackage[T1]{fontenc}                            % Codificação da fonte em 8 bits
\usepackage{graphicx}                               % Inserir figuras
\usepackage{amsfonts, amssymb, amsmath}             % Fonte e símbolos matemáticos
\usepackage{booktabs}                               % Comandos para tabelas
\usepackage{verbatim}                               % Texto é interpretado como escrito no documento
\usepackage{multirow, array}                        % Múltiplas linhas e colunas em tabelas
\usepackage{indentfirst}                            % Endenta o primeiro parágrafo de cada seção.
\usepackage{listings}                               % Utilizar codigo fonte no documento
\usepackage{xcolor}
\usepackage{microtype}                              % Para melhorias de justificação?
\usepackage[portuguese,ruled,lined]{algorithm2e}    % Escrever algoritmos
\usepackage{algorithmic}                            % Criar Algoritmos  
\usepackage{amsgen}
\usepackage{lipsum}                                 % Usar a simulação de texto Lorem Ipsum
\usepackage{tocloft}                                % Permite alterar a formatação do Sumário
\usepackage{etoolbox}                               % Usado para alterar a fonte da Section no Sumário
\usepackage[style=index,noredefwarn,nogroupskip,nonumberlist,acronym,automake]{glossaries}                % Permite fazer o glossario
\usepackage{caption}                                % Altera o comportamento da tag caption
\usepackage{cite}
\usepackage[alf, abnt-emphasize=bf, bibjustif, recuo=0cm, abnt-etal-cite=2, abnt-etal-list=0]{abntex2cite}  % Citações padrão ABNT
\usepackage{mathptmx}                               % Usa a fonte Times New Roman										
\usepackage{lscape}                                % Permite páginas em modo "paisagem"
\usepackage{appendix}                               % Gerar o apendice no final do documento
\usepackage{paracol}                                % Criar paragrafos sem identacao
%\usepackage{lib/uecetex2}		                    % Biblioteca com as normas da UECE para trabalhos academicos
\usepackage{pdfpages}                               % Incluir pdf no documento
\usepackage{amsmath}                                % Usar equacoes matematicas
\usepackage{courier}
\usepackage{setspace}
\usepackage{acronym}
\usepackage{lib/tccfurg/maintcc}
\usepackage{lastpage}
\usepackage{float}
\usepackage{siunitx}
\makeglossaries
\makeindex
