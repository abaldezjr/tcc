%PACOTES

% compilação de fontes
\usepackage{mathtools}
\usepackage{amsfonts, amssymb, amsmath}             % Fonte e símbolos matemáticos
\usepackage[utf8]{inputenc}
\usepackage[T1]{fontenc}

%layouts
\usepackage{appendix}                               % Gerar o apendice no final do documento
%\usepackage[nogroupskip,nonumberlist,acronym]{glossaries}                % Permite fazer o glossario
\usepackage{indentfirst}                            % Identa o primeiro parágrafo de cada seção.
\usepackage{multicol}                               % permite trabalhar em mais de uma coluna
\usepackage{multirow, array}                        % Múltiplas linhas e colunas em tabelas

%outros pacotes úteis e específicos
\usepackage[alf, abnt-emphasize=bf, bibjustif, recuo=0cm, abnt-etal-cite=2, abnt-etal-list=0]{abntex2cite}  % Citações padrão ABNT
\usepackage[americanvoltages]{circuitikz}           % desenhar circuitos
\usepackage[portuguese,ruled,lined]{algorithm2e}    % Escrever algoritmos
\usepackage{algorithmic}                            % Criar Algoritmos  
\usepackage{amsgen}
\usepackage{booktabs}                               % Comandos para tabelas
\usepackage{caption}                                % Altera o comportamento da tag caption
\usepackage{charter}                                % pacote para a fonte do texto
\usepackage{icomma}                                 % separador de decimal
\usepackage[utf8]{inputenc}                         % Acentuação direta
\usepackage{etoolbox}                               % Usado para alterar a fonte da Section no Sumário
\usepackage[T1]{fontenc}                            % Codificação da fonte em 8 bits
\usepackage{float}                                  % permite escolher o local da imagem
\usepackage{graphicx}                               % Inserir figuras
\usepackage{hyperref}                               % permite criar hyperlinks de autorreferências dentro do texto
\usepackage{lmodern}                                % latin modern
\usepackage{longtable}
\usepackage{lipsum}                                 % Usar a simulação de texto Lorem Ipsum
\usepackage{listings}                               % Utilizar codigo fonte no documento
\usepackage{microtype} 				                % para melhorias de justificação
\usepackage{natbib}
\usepackage{notoccite}                              % impede que uma referência da lista de figuras seja contada como referência do trabalho, mudando a ordem com que elas são apresentadas
\usepackage{paracol}                                % Criar paragrafos sem identacao
\usepackage{pdfpages}                               % Incluir pdf no documento
\usepackage{siunitx}                                % algum pacote pra unidades do SI acho que nem uso?
\usepackage{subcaption}                             % permite criar sublegendas em figuras (a),(b),(c), etc.
\usepackage{tablefootnote}                          % permite criar uma nota de rodapé em tabela
\usepackage{tocloft}                                % Permite alterar a formatação do Sumário
\usepackage{tikz}                                   % pacote pra desenho
\usepackage{titlesec}                               % customização de seções e capítulos
\usepackage{units}                                  % escreve unidades em romano dentro do modo matemático
\usepackage{verbatim}                               % Texto é interpretado como escrito no documento
\usepackage{xcolor}
\usepackage{mathptmx}                               % Usa a fonte Times New Roman										

\usepackage{lib/variaveis}
\usepackage{lib/funcoes}

\setlength{\columnsep}{1cm} %define separação entre as colunas
\hypersetup{
	colorlinks=true, %links coloridos
	linkcolor=blue, %cor dos links
	citecolor=red % cor das citações
}

\counterwithin{figure}{chapter}                     % contar figuras por capítulo
\counterwithin{table}{chapter}                      % conta as tabelas por capítulo
